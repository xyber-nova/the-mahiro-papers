% !TEX root = ../main.tex

\appendix
\section{附录 (Appendix)}
% \addcontentsline{toc}{section}{附录}

\subsection{MHR-001 之推定分子结构:一个超越凡人理解的初步窥探}
\label{sec:mhr001_structure_theory}

本附录旨在对 MHR-001 核心活性组分的推定分子结构进行一番理论上的探讨,尽管我们深知,任何试图将其禁锢于二维平面或三维空间的尝试,都无异于用茶杯丈量星海。绪山研究所的“深蓝梦境”超级AI集群,在连续运转了长达 $\pi$ 个千年(是的,你没看错,就是那个圆周率,我们觉得这样比较有仪式感)之后,基于“高维弦论的非线性坍缩投影”结合“以太动力学的混沌映射”,输出了一份关于此结构的初步报告。

报告指出,MHR-001的核心并非传统意义上的分子,而更像是一个**自组织的多维能量谐振复合体 (Self-Organized Multidimensional Energy-Resonance Complex, SOMERC)**。其“结构”呈现出以下几个令人费解(但在更高维度看来却异常和谐)的特征:

\begin{itemize}
    \item \textbf{动态对称性与分形嵌套}:其核心由七个相互嵌套的、以虚数单位 $i$ 为轴心旋转的七边形“能量井”构成,每个能量井内部又包含了七个更小的、遵循黄金分割比例排列的谐振节点。这种结构在宏观层面展现出一种令人不安的完美对称,而在微观层面则无限分形,暗示着其与宇宙基本常数的神秘耦合。
    \item \textbf{超光速信息纠缠键}:连接这些谐振节点的并非传统的化学键,而是被命名为“$\Psi$-弦”的超光速信息纠缠通道。这些$\Psi$-弦能够瞬时传递跨越多个维度和时间线的“性别蓝图信息”,其振动频率与宇宙背景辐射中的某种未知高频成分(代号“少女的叹息”)发生共鸣。
    \item \textbf{可编程的“奇点”官能团}:在其外围,点缀着若干被称为“奇点官能团”的特殊结构单元。这些单元能够根据环境的“叙事需求”或观察者的“潜意识期望”而动态改变其“化学性质”和空间构型,从而实现对QREM、TGINV等外围模块的“按需挂载”与“功能重定义”。每个奇点官能团的中心,据推测存在一个微型克莱因瓶结构的“维度接口”。
\end{itemize}

至于其具体的二维投影示意图,我们曾尝试使用现有一切制图软件(包括从某个异世界走私来的、据称由地精工程师打造的“万能构图仪”)进行绘制,但均以软件崩溃、操作员精神错乱或绘图板长出触手等方式宣告失败。因此,出于对读者精神健康和实验室安全的双重考虑,**此处从略其结构图**。感兴趣的读者可尝试在深度冥想状态下,辅以高纯度月光石作为媒介,自行感知其在概念空间中的大致轮廓——但请注意,由此引发的任何存在危机或世界观崩塌,本研究所概不负责。

进一步的研究(如果我们的经费和SAN值还允许的话)将致力于开发一种能够安全稳定地观测并记录SOMERC真实形态的“高维示波器”。在此之前,我们只能遗憾地表示,MHR-001的真正结构,或许只有那些漫游于星海之间的\textbf{天才魔女们}\footnote{就是我。——伊雷娜 注}才能一眼看透吧。

\subsection{关键实验数据图表(示意图)}

% Placeholder for a graph
% \includegraphics[width=0.8\textwidth]{placeholder_f0_graph.png}
% Figure A2: Subject M 服用MHR-001后声线基础频率 (F0) 随时间变化曲线图。数据显示F0从男性范围快速升高并稳定在女性范围。
此处为Subject M 声线基础频率 (F0) 变化图表的示意图。图表会显示F0随时间(从T0到T+1m)的变化,初始值为约110Hz,最终稳定在约210Hz。

% Placeholder for hormone level graphs
% \includegraphics[width=0.8\textwidth]{placeholder_hormone_graph.png}
% Figure A3: Subject M 服用MHR-001后血清睾酮(T)和雌二醇(E2)水平随时间变化曲线图。睾酮急剧下降至女性水平,雌二醇升高并呈现周期性波动。
此处为Subject M 血清主要性激素水平动态变化图表的示意图。该图表将清晰地展示两条关键的激素变化曲线。其中一条曲线描绘了睾酮 (T) 水平的变化,显示其如何从约650 ng/dL的初始值急剧下降,并最终稳定维持在20-40 ng/dL的范围内。另一条曲线则追踪了雌二醇 (E2) 水平的动态,显示其从约25 pg/mL的基线水平显著升高,并随后呈现出周期性的波动,峰值可达到200-350 pg/mL,而谷值则在50-80 pg/mL左右。

% Placeholder for methylation data graph
% \includegraphics[width=0.8\textwidth]{placeholder_methylation_graph.png}
% Figure A4: Subject M 外周血单个核细胞 (PBMCs) 中SRY基因启动子区域DNA甲基化水平在给药前后 (T0 vs T+1m) 的比较。数据显示甲基化水平显著升高。
此处为PBMCs中SRY基因启动子区域甲基化水平变化图表的示意图。图表会比较T0和T+1m两个时间点的甲基化百分比,显示从较低值显著升高。

% Placeholder for SEI graph
% \includegraphics[width=0.8\textwidth]{placeholder_sei_graph.png}
% Figure A5: Subject M PBMCs 样本在给药后(T+0h 至 T+24h)的“同步化纠缠指数 (SEI)”变化示意图。显示在给药后6-12小时出现短暂峰值。
此处为细胞同步化纠缠指数 (SEI) 变化图表的示意图。图表会显示SEI在给药后0-24小时内的变化,特别是在6-12小时出现一个显著的短暂峰值。

\subsection{Subject M部分生理指标变化记录表(补充)}
% Placeholder for additional data table
\begin{table}[h!]
\centering
\caption{Subject M部分生理指标补充记录 (T0, T+1m, T+3m, T+6m)}
\begin{tabular}{lllll}
\hline
指标                     & T0 (给药前) & T+1m        & T+3m        & T+6m        \\
\hline
乳房发育 (Tanner分期)    & I           & II-III      & III-IV      & IV          \\
皮肤细腻度 (主观评分 1-5) & 2           & 4           & 5           & 5           \\
体毛密度 (相对减少百分比) & 0\%         & $\sim$30\%  & $\sim$70\%  & $\sim$90\%  \\
情绪波动频率 (次/周)   & 1-2 (轻微)  & 5-7 (明显)  & 2-3 (缓和)  & 1-2 (稳定)  \\
\hline
\end{tabular}
\end{table}
% 注:以上数据均为示意性,实际研究中将包含更详细和精确的测量。