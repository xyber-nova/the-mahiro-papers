% !TEX root = ../main.tex

\section*{致谢 (Acknowledgements)}
\addcontentsline{toc}{section}{致谢} % Add to ToC manually if needed for unnumbered chapters

首先,由衷感谢东都综合大学先端生命科学研究所那台服役多年、脾气古怪但总能在关键时刻吐出咖啡的咖啡机,没有它的鼎力支持,本研究可能在无数个不眠之夜中途夭折。同时,感谢研究所英明神武的领导(如果他们能批更多经费就更好了)和那些在背后默默吐槽但从未放弃我们的(虚构的)导师团队。

本研究的核心,也是最大的“惊喜”来源——实验对象Subject M(绪山真寻君,一位在科学史上留下了浓墨重彩一笔的普通(前)男性),必须得到最隆重的感谢。感谢他“自愿”贡献了其宝贵的生理数据、情绪波动日常、以及对未知命运的(大部分时间里的)坦然接受。没有他的“牺牲”(研究者语)与“配合”(在其无法反抗的前提下),这项伟大的研究将无从谈起。愿他在新的身份下,游戏打得更开心,手办买得更顺心。

特别鸣谢绪山家的列祖列宗,感谢他们优秀的遗传基因,为本研究提供了如此理想的(且易于获取的)实验样本。

感谢穗月枫女士、穗月椛女士等热心邻居,她们在浑然不知情的情况下,为本研究的“社会行为学观察”阶段提供了丰富、真实且极具价值的“社交刺激源”和“对照组”环境。她们的日常八卦与友好互动,为我们观察Subject M转变后的社交适应性提供了绝佳素材。

还要感谢那些启发了本研究核心理论的科幻小说作家、漫画家以及不愿意透露姓名的动漫角色们,是他们的奇思妙想点燃了科学的火花(尽管有时也可能点燃了实验室)。

最后,感谢所有打印机、墨盒、A4纸以及在无数次修改中幸存下来的键盘,你们也是这项研究不可或缺的英雄。以及,感谢看到这里的读者,希望这篇论文没有对您的世界观造成过大的冲击。