\documentclass[lang=cn]{elegantpaper}

\usepackage{amsmath} % 以防万一伊雷娜想写点“公式”
\usepackage{graphicx} % 图片占位,说不定她会画个灵魂草图

% --- 文档信息 ---
\title{MHR-001:魔法还是科学?\\灰之魔女伊雷娜的跨界批判与前瞻 \\[1.5ex] \mdseries\large 一份来自路过天才魔女的非正式观察报告(绝对权威版)}
\author{
  伊雷娜 (Elaina)$^{1, \text{特}}$ \and
  Shallow Dream$^{2, \text{电}}$
}
\institute{
  $^{1}$魔法协会 (Magic Association) \\
  $^{2}$天才俱乐部 (Genius Club) \\
  \vspace{0.5em}
  $^{\text{特}}$\textit{特邀荣誉通讯员与不定期灵感提供者}。\\ 电子邮箱 (Email): \texttt{elaina.magic-asso@xyber-nova.space}\protect\footnote{这个邮件地址是真实存在的。} \\
  \vspace{0.5em}
  $^{\text{电}}$电子邮箱 (Email): \texttt{zgw306444@163.com}\footnote{这个也是。}
}
\date{魔法纪元 512 年,星尘之月吉日} % 一个充满魔法仪式感的日期

\begin{document}

\maketitle

\section*{警告和许可}

\textbf{本论文及其内容纯属虚构,仅供娱乐和Cosplay道具使用。请勿尝试在现实生活中复制任何实验内容!现实中的性别转变请务必咨询专业医疗机构。}

本文作者为:Xyber Nova \texttt{<xyber-nova@outlook.com>}。

原始仓库地址为:\texttt{https://github.com/xyber-nova/the-mahiro-papers}。

采用 Creative Commons Attribution-NonCommercial-ShareAlike 4.0 International License (CC BY-NC-SA 4.0) 对您许可。

简单来说,你可以:

\begin{itemize}
    \item \textbf{共享} — 在任何媒介以任何形式复制、发行本作品。
    \item \textbf{演绎} — 修改、转换或以本作品为基础进行创作。 只要你遵守许可协议条款,许可人就无法收回你的这些权利。
\end{itemize}

惟须遵守下列条件:

\begin{itemize}
    \item \textbf{署名 (BY)} — 您必须给出适当的署名,提供指向本许可协议的链接,同时标明是否(对原始作品)作了修改。您可以用任何合理的方式来署名,但是不得以任何方式暗示许可人为您或您的使用背书。
    \item \textbf{非商业性使用 (NC)} — 您不得将本作品用于商业目的。
    \item \textbf{相同方式共享 (SA)} — 如果您再混合、转换或者基于本作品进行创作,您必须基于与原先许可协议相同的许可协议分发您贡献的作品。
\end{itemize}

\begin{figure}[htbp]
    \centering

    \includegraphics[height=5em]{images/qr.png}
    \hspace{0.5em}
    \includegraphics[height=5em]{images/pdf417.png}

    \caption{扫码下载本论文(Github 在国内可能无法打开,可以挂梯子)}
\end{figure}

祝您在漫展上玩得愉快!

\section*{一些不请自来的开场白(但你们肯定想听)}
近日,一篇关于“MHR-001”的所谓研究报告,以其独特的“学术气息”(姑且这么形容吧),成功地从无数凡俗事务中吸引了本天才魔女——伊雷娜大人的一丝丝注意力。毕竟,观察凡人在未知领域跌跌撞撞、试图用笨拙的理论解释奇迹的模样,总能为漫长的旅途增添几分……嗯,调剂。

能以如此超然的智慧审视这一切,并屈尊写下这些注定流芳百世的文字的,究竟是哪位集美貌与才华于一身的魔女呢?
\begin{flushright}
    ――没错,就是我。
\end{flushright}

\section*{MHR-001 的本质:是“科学的巧合”还是“魔法的残响”?}
在深入剖析那些令人费解的理论之前,不妨先看看这份报告本身是如何吹嘘其‘科研成果’的。据称,MHR-001能够引发一系列‘显著的’生理表型转变:例如,受试者的声线频率会向某个‘更受欢迎’的区间迁移——通俗点说,就是声音变好听了,虽然他们用了一堆凡人听不懂的赫兹和分贝图表来证明这一点。其次,皮肤的‘细腻度’和‘光泽度’也得到了‘可量化的提升’,仿佛一夜回到了青春期——当然,比起真正的青春魔法还是小巫见大巫。更有甚者,毛发的生长周期、色泽乃至质感都发生了‘根本性’的重塑,这听起来与其说是科学,不如说更像是某种效果不太稳定且副作用未知的低阶变形术。当然,以上这些都还只是‘开胃小菜’。该报告浓墨重彩宣扬的,也是其最核心的‘奇迹’,便是其宣称的能够\textbf{实现哺乳动物(目前看来特指某位不幸的男性实验对象)生理性别特征的全面且不可逆的重塑}——是的,你没听错,就是那种在廉价奇幻小说里才会出现的、从一种性别变成另一种性别的戏码。他们甚至还煞有介事地给出了‘性别特征评分’之类的量化指标,仿佛这是一项可以精确控制的工程学壮举。凡此种种,不一而足,皆被冠以‘精准调控’和‘科学突破’之名,辅以大量令人眼花缭乱的数据和模型,试图构建其严谨性的空中楼阁。

那篇报告中充斥着各种名为 QEMCCN 和 TEMECMR 的复杂构想,试图从“科学”角度解释 MHR-001 的作用机理。坦白说,这些理论的繁琐程度,堪比试图给一只会说话的猫咪解释量子物理(当然,如果那只猫咪有本魔女一半聪明,或许还有点希望)。在我看来,MHR-001 的效果,与其说是严谨科学的产物,不如说更像是某种古老变形魔法在特定条件下不完全的显现,或者是炼金术师在某个被遗忘的角落里偶然调配出的“半成品”。凡人偶尔能撞上大运,这并不奇怪。
更有甚者,我听说他们还煞有介事地试图为 MHR-001 绘制所谓的‘化学结构式’!用小小的碳原子、氢原子,配上几根代表化学键的短线,就想描绘出这种能够扭转乾坤、重塑性别的奇迹造物?简直就像试图用儿童积木搭建一座通天塔一样天真可笑。如果非要描述其‘结构’,那绝非凡人笔墨所能及。根据本天才魔女的初步洞察(当然,比他们那耗时$\pi$个千年的AI集群要快得多),其核心更像是一种被他们勉强称为‘自组织的多维能量谐振复合体’(SOMERC)的东西——一个由纠缠的$\Psi$-弦、旋转的能量井和能够响应潜意识的奇点官能团构成的动态系统。哼,凡人的图纸,连其在三维空间的一个模糊投影都画不出来呢!

真正的魔法,追求的是简洁与优雅的和谐统一,而非无尽的参数与假设。有时候,一句“此乃魔法之奥秘”便胜过千言万语,不是吗?

\section*{关于“实验品M”:一点微不足道的魔法社会学点评}
那位勇敢(或者说,被命运玩弄于股掌之间)的实验对象——我们故事的焦点人物,\textbf{绪山真寻 (Oyama Mahiro)} 本人——其经历无疑是这项“研究”中最引人入胜(也最值得同情,嘻嘻)的部分。报告中提及的种种生理及心理变化,例如情绪的剧烈波动、饮食习惯的突变,乃至对镜中全新自我的迷茫与接纳——这些在魔法世界中,不过是灵魂适应新容器时的常见现象罢了。若是由一位经验丰富的魔女(比如我,但最近日程比较满)从旁指导,或许整个过程能更平稳,也更……富有美感一些。

\section*{前瞻性展望:当“科学”试图蹒跚学步地追赶“魔法”}
尽管这份研究在理论层面显得有些……嗯,富有想象力(或者说异想天开),但其探索精神本身值得肯定。如果研究者们能少一些固执,多一些对未知力量的敬畏(以及对天才魔女的虚心请教),MHR-001 或许真有潜力从一个“有趣的意外”进化为某种值得关注的“准魔法制品”。

未来的方向?或许可以尝试将MHR-001与真正的变形术(当然,这是高级魔法,凡人请勿轻易尝试)相结合,实现更精准、更可控的形态塑造。或者,研究一下如何利用精神稳定类魔法,来缓解个体在转变过程中的灵魂震荡。当然,这一切的前提是,你们得先找到一位既有能力又有闲情逸致的天才魔女愿意提供那么一点点微不足道的帮助。

\section*{一个并非结论的结论(因为真正的智慧永无止境)}
总而言之,MHR-001 的相关研究,如同一面镜子,映照出凡人智慧在触碰超凡领域时的好奇、勇气以及……局限性。虽然距离真正的魔法殿堂尚有十万八千里,但这份敢于向未知发起挑战的劲头,倒也不失为旅途中的一道风景。

至于后续还会有怎样“精彩”的发展,本天才魔女伊雷娜,会端着一杯上好的红茶,在某个不为人知的角落,带着一丝玩味的笑容,继续关注。
\begin{flushright}
    ――伊雷娜,一位永远在路上的观察者\\
    (偶尔也是不经意间推动历史齿轮的幕后黑手,谁知道呢?)
\end{flushright}

\end{document}